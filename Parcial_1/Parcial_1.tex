\documentclass[10.5pt]{article}

% Spanish characters
\usepackage[utf8]{inputenc}
\usepackage[T1]{fontenc}
% French display
\usepackage[english,spanish]{babel}

\usepackage{lastpage}
%Esto me permite usar el comando "\pageref{LastPage}" en el footer.
\renewcommand{\baselinestretch}{1.6}
% Esto controla el interlineado o espaciado!!!
\usepackage{color}
%\newcommand{\red}[1]{{\color{red} #1}}
\newcommand{\red}[1]{{\color{black} #1}}

%Esto me permite poner hipervínculos:
%\usepackage[pdftex,
%       colorlinks=true,
%       urlcolor=blue,       % \href{...}{...} external (URL)
%       filecolor=green,     % \href{...} local file
%       linkcolor=black,       % \ref{...} and \pageref{...}
%       pdftitle={Papers by AUTHOR},
%       pdfauthor={Your Name},
%       pdfsubject={Just a test},
%       pdfkeywords={test testing testable},
%%       pagebackref,%Esto parece que pone un numerito al lado de la referencia (en la seccion de bibliografia), donde se puede clicar y te lleva al lugar del texto donde se le cita.
%       pdfpagemode=None,
%       bookmarksopen=true]{hyperref}


%The following packages are relics, but I don't want to remove them just in case:
\usepackage{amsmath}
\usepackage{array}
\usepackage{latexsym}
\usepackage{amsfonts}
\usepackage{amsthm}
\usepackage{cite}
\usepackage{setspace}
\usepackage{amssymb}
\usepackage{hyperref}

\usepackage{multicol}
\usepackage{color}
%\usepackage{minipage}

\usepackage{graphicx} % Required for including images
\graphicspath{{figures/}} % Location of the graphics files
\usepackage[font=small,labelfont=bf]{caption} % Required for specifying captions to tables and figures

%The defaults margins are huge, so I'll customize it:
\oddsidemargin  -0.0 in
\textwidth 6.5 in
\textheight 8.7 in
\addtolength{\voffset}{-1cm}

%%%%%%%%%%%%%%%%%%%%%%%% HEADER AND FOOTER %%%%%%%%%%%%%%%%%%%%
\usepackage{fancyhdr}
\pagestyle{fancy}

\fancyhead[L]{Lecci\'{o}n 1}
%\fancyhead[L]{CNRS Competition 01-04}
\fancyhead[R]{Jos\'{e} David Ruiz \'{A}lvarez}
\fancyhead[C]{}
\fancyfoot[C]{\thepage /\pageref{LastPage}}

\newlength\FHoffset
\setlength\FHoffset{0cm}

\addtolength\headwidth{2\FHoffset}
\fancyheadoffset{\FHoffset}

\newlength\FHleft
\newlength\FHright

\setlength\FHleft{1cm}
\setlength\FHright{1cm}

\thispagestyle{empty}
%%%%%%%%%%%%%%%%%%%%%%%% HEADER AND FOOTER %%%%%%%%%%%%%%%%%%%%



\begin{document}

%\begin{center}
\noindent
\begin{minipage}[b]{0.75\linewidth}
{\LARGE\bf Parcial 1}\\ %[1mm]
%\end{center}
%{\Large\bf \emph{}}\\ %[3mm]
%{\Large\bf \emph{connections between LHC and neutrino experiments}}
\large{Jos\'{e} David Ruiz \'{A}lvarez} \\
\small{\href{mailto:josed.ruiz@udea.edu.co}{josed.ruiz@udea.edu.co}} \\ %[3mm]
%\normalsize{Plaza código: 2017010307, Área: Física de fenomenología de altas energías} \\%[3mm]
\normalsize{Instituto de Física, Facultad de Ciencias Exactas y Naturales} \\%[3mm]
\normalsize{\bf Universidad de Antioquia} \\[8mm]
\today %\\[4mm]
\end{minipage}%
%\end{center}
%\begin{minipage}[b]{0.25\linewidth}
%\centering{\includegraphics[width=4cm]{figures/CMS.png}}\\
%%%%%\includegraphics[width=15cm]{figures/UniandesColombia.jpg}\\
%\end{minipage}

%\begin{center}
%{\bf Palabras clave:} CERN, LHC, CMS, Materia Oscura
%\end{center}

%\doublespacing

\section{Contenido}

El siguiente problema es para ser desarrollado en grupo de m\'{a}ximo dos personas. Grupos formados presencialmente el 19 de ferbrero en hora de clase, de lo contrario se trabaja de forma individual. El lenguaje de programaci\'{o}n de para resolver el problema es de libre elecci\'{o}n, sin embargo la utilizaci\'{o}n de C++ y ROOT da un sobrepuntaje de 0.5 unidades.

\section{Planteamiento}

Considere el juego de la vida con sus reglas tradicionales. Considere un matriz cuadrada de dimensiones $n \times n$ en donde cada uno de sus elementos puede asumir dos valores 0 o 1. Dicha matriz evoluciona temporalmente a trav\'{e}s de iteraciones de acuerdo a las siguiente reglas:

\begin{itemize}
\item Si un elemento tiene como valor 0 y tiene 3 vecinos con valor 1, transforma su valor a 1.
\item Si un elemento tiene como valor 1 y tiene 2 o 3 vecinos con valor 1, conserva su valor de 1. De lo contrario, transforma su valor a cero.
\end{itemize}

\section{Problema}

Para el desarrollo del problema necesita como base un script que genere una matriz con estado inicial aleatorio y la haga evolucionar de acuerdo a las reglas del juego de la vida descritas en el planteamiento. Con ese script como base desarrolle las siguientes tareas:

\begin{itemize}
\item Modifique el script para que reciba por entrada de l\'{i}nea de comandos el valor $n$ que fija las dimensiones de la matriz. Solo considere matrices cuadradas. (Valor=0.5)
\item Modifique el script para que se detenga cuando el n\'{u}mero de celdas con valor 1 se estabiliza (deja de cambiar). (Valor=0.5)
\item Tome datos del tiempo que le toma al sistema estabilizarse para matrices de dimensi\'{o}n entre $n=5$ y $n=60$. Cada valor de $n$ debe tener al menos 20 datos. Debe tomar al menos 12 valores de n. (Valor=1.0)
\item Calcule el tiempo m\'{a}ximo de estabilizaci\'{o}n del sistema para cada valor de $n$. Hint: Puede utilizar el comando $time$ de linux, tomando el tiempo de procesamiento del sistema. (Valor=0.2)
\item Ajuste los datos correspondientes a cada n con una gaussiana y calcule el valor central de dicha distribuci\'{o}n. (Valor=0.8)
\item Encuentre una funci\'{o}n que ajuste el valor m\'{a}ximo de estabilizaci\'{o}n del sistema como funci\'{o}n de n. (Valor=1.0)
\item Encuentre una funci\'{o}n que ajuste el valor medio de estabilizaci\'{o}n del sistema obtenido a trav\'{e}s del fit a una gaussiana como funci\'{o}n de $n$. (Valor=1.0)
\end{itemize}

{\bf Entregables}: Script con modificaciones. Conjunto de datos (COMPLETO). Scripts/notebooks que hacen los ajustes de los tres \'{u}ltimos \'{i}tems. Gr\'{a}ficas y dem\'{a}s material que considere pertinente.

{\bf M\'{e}todo de entrega}: Pull request al repositorio central del curso con una carpeta con los nombres de los integrantes de cada curso que debe estar adentro de la carpeta llamada ``Parcial\_1'' con todos los entregables.

\end{document}

%%% Local Variables:
%%%   mode: latex
%%%   mode: flyspell
%%%   ispell-local-dictionary: "spanish"
%%% End:
